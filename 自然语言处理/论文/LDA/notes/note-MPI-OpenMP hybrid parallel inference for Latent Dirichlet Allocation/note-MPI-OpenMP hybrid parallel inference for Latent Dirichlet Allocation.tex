\documentclass{article}
\usepackage{indentfirst}
\usepackage{ctex}
\usepackage{geometry}
\usepackage{enumerate}
\geometry{left=3.17cm,right=3.17cm,top=2.54cm,bottom=2.54cm} % 页边距

\begin{document}

\title{MPI/OpenMP hybrid parallel inference for Latent Dirichlet Allocation}
\date{}
\maketitle

\begin{abstract}
用于LDA参数估计的变分贝叶斯推断和collaopsed Gibbs sampling在处理大规模数据时计算开销非常大。本文利用并行计算技术提高Gibbs sampling推断的效率。我们使用一种近年来被广泛使用的共享内存集群(SMP集群)。在LDA并行推理的前期工作中,MPI和OpenMP都以被使用过。另一方面,对于SMP集群而言,更适合采用在SMP节点和循环指令之间通过消息传递来实现通信的混合并行化,以此在每个SMP节点之间实现并行化。本文设计了一种用于LDA的MPI/OpenMP混合并行推断方法。
\end{abstract}

\section{Introduction}
针对大量文档集的分析中,主题模型方法是一种极为成功的学习算法。主题模型基于的思想是,每篇文档都是从单词分布的“混合”中生成的,每一个这样的“混合”称为“主题”。
...

\section{Related Work}
\subsection{LDA}

\subsection{Fast inference methods for LDA}
collapsed Gibbs sampling的计算复杂性来自于文档集中的主题数量和词汇表大小的乘积。针对此问题的改进工作有很多:
\begin{itemize}
	\item ...
	\item ...
\end{itemize}

\end{document}
 
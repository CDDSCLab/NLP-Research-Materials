\documentclass{article}
\usepackage{indentfirst}
\usepackage{ctex}
\usepackage{geometry}
\usepackage{enumerate}
\geometry{left=3.17cm,right=3.17cm,top=2.54cm,bottom=2.54cm} % 页边距

\begin{document}

\title{Clustering with Probabilistic Topic Models on Arabic Texts: A Comparative Study of LDA and K-Means}
\date{}
\maketitle

\section{Introduction}
由于阿拉伯语的特殊性,针对该语言文档的聚类所面临的困难也是特殊的。本文比较了阿拉伯语的特殊性对LDA和K-means的性能影响。

\section{Document Clustering}
\subsection*{LDA and Clustering}
根据[16],\textbf{使用主题模型进行文档聚类有两种方法}。第一种方法使用主题模型对文档进行降维表示(单词表示$\rightarrow$主题表示),然后在新的表示形式上应用标准聚类算法(比如K-means);而另一种方法直接使用主题模型,其思想是,在完成参数$\phi$和$\theta$的估计后,每个主题$z$就变成了一个新的聚类,被分配给这个聚类的文档在所有分配给主题$z$的文档中拥有最大概率。\par
本文使用的是第二种方法,该方法允许我们测量LDA相比于传统聚类算法(如K-means)的性能。

\end{document}
 
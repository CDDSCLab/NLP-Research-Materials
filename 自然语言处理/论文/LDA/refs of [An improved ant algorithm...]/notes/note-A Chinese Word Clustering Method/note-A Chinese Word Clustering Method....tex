\documentclass{article}
\usepackage{indentfirst}
\usepackage{ctex}
\usepackage{geometry}
\usepackage{enumerate}
\geometry{left=3.17cm,right=3.17cm,top=2.54cm,bottom=2.54cm} % 页边距

\begin{document}

\title{A Chinese Word Clustering Method Using Latent Dirichlet Allocation and K-means}
\date{}
\maketitle

\section{Introduction}
本文提出了一种基于LDA和K-means的中文单词聚类方法。K-means作为一种无监督聚类算法,由于初始聚类中心是随机选择的,所以得到的聚类结果不唯一。为了解决词问题,我们通过LDA算法从每个句子中的名词中抽取主题,然后选择每个主题下概率最高的名词作为K-means的初始聚类中心,最后使用K-means对文本中的所有单词进行聚类。


\end{document}
 